\documentclass[conference]{IEEEtran}
\newcommand*{\rootPath}{../}
\usepackage{url}

% math and cs
% \usepackage[]{algorithm2e}
\usepackage[linesnumbered,lined,boxed,commentsnumbered]{algorithm2e}
\usepackage{amsmath}
\usepackage{amssymb}
\usepackage{mathrsfs}
\newtheorem{definition}{Definition}
\newtheorem{proposition}{Proposition}
\newenvironment{proof}[1][Proof]{\begin{trivlist}
  \item[\hskip \labelsep {\bfseries #1}]}{\end{trivlist}}
  % \newenvironment{definition}[1][Definition]{\begin{trivlist}
  % \item[\hskip \labelsep {\bfseries #1}]}{\end{trivlist}}
  % \newenvironment{example}[1][Example]{\begin{trivlist}
  % \item[\hskip \labelsep {\bfseries #1}]}{\end{trivlist}}
  % \newenvironment{remark}[1][Remark]{\begin{trivlist}
  % \item[\hskip \labelsep {\bfseries #1}]}{\end{trivlist}}

% style
\usepackage{booktabs}
\usepackage{multirow}
\usepackage{lipsum}
\usepackage{todonotes}
\usepackage{standalone}
\usepackage{import}
\usepackage{url}
%\Urlmuskip=0mu plus 1mu %bug url

% graph
\usepackage{graphicx}
\usepackage[outdir=./]{epstopdf}
\usepackage[labelformat=simple]{subcaption}
\usepackage{array}
%\usepackage[colorinlistoftodos]{todonotes}
\newcommand{\HRule}{\rule{\linewidth}{0.5mm}}



\DeclareCaptionLabelSeparator{periodspace}{.\quad}
\captionsetup{font=footnotesize,labelsep=periodspace,singlelinecheck=false}
\captionsetup[sub]{font=footnotesize,singlelinecheck=true}


\usepackage[english,american]{babel}



\usepackage[capitalise,nameinlink]{cleveref}
%Nice formats for \cref
\crefname{section}{Sect.}{Sect.}
\Crefname{section}{Section}{Sections}
\crefname{figure}{Fig.}{Fig.}
\Crefname{figure}{Figure}{Figures}

\usepackage{xspace}
%\newcommand{\eg}{e.\,g.\xspace}
%\newcommand{\ie}{i.\,e.\xspace}
\newcommand{\eg}{e.\,g.,\ }
\newcommand{\ie}{i.\,e.,\ }


\renewcommand\thesubfigure{(\alph{subfigure})}


%invert table
\usepackage{collcell}
\usepackage{datatool}
\usepackage{environ}


\standalonetrue

\begin{document}

%%=============================================================================



\section{Le jugement majoritaire}
\label{sec:mj}

Le jugement majoritaire (JM) a \'et\'e developp\'e par Balinski et Laraki \cite{mj} pour proposer une alternative au vote majoritaire. Nous pr\'esentons succinctement son fonctionnement, ses avantages et inconv\'enients. Pour plus de d\'etails, le lecteur est invit\'e \`a lire les articles mis en r\'ef\'erence.


\subsection{Fonctionnement}

Le JM demande \`a chaque \'electeur de juger chaque candidat selon au moins 5 mentions: Tr\`es bien, Bien, Assez bien, Correct et D\'efavorable. Lorsque tous les jugements ont \'et\'e exprim\'es, la m\'edianne des jugements de chaque candidat est calcul\'ee. En partant du meilleur jugement (ou du pire, a verifier), la m\'edianne est la mention pour laquelle le nombre de mentions sup\'erieures est \'egal au nombre de mentions inf\'erieures. Par exemple, si un candidat A a les mentions: Tr\`es bien, Tr\`es bien, Assez bien, Correct et D\'efavorable, son r\'esultat sera Assez bien. 

En cas d'\'egalit\'e entre deux candidats A et B, une proc\'edure appel\'ee \emph{algorithme de tie-breaking} permet de trouver le gagnant entre les deux candidats. Cette proc\'edure consiste \`a retirer les mentions m\'ediannes de A et B jusqu'\`a ce qu'elles diff\`erent. Par exemple, si le candidat B a les mentions: Tr\`es bien, Assez bien, Assez bien, Correct et D\'efavorable, alors A et B ont les m\^emes mentions m\'ediannes Assez bien. On retire alors momentan\'ement cette mention pour d'\'epartager A et B. A a alors les mentions: Tr\`es bien, Tr\`es bien, Correct et D\'efavorable. Sa mention m\'edianne est d\'esormais Tr\'es bien. En revanche, B a les mentions Tr\`es bien, Assez bien, Correct et D\'efavorable. Sa mention m\'edianne reste Assez bien. C'est pourquoi, B est class\'e apr\'es A. Si B avait toujours la m\^eme mention que A, on aurait continu\'e la proc\'edure en retirant une nouvelle fois (et jusqu'\`a ce que A se d\'epartage de B) la mention m\'edianne.

\subsection{Avantages}

Cette mani\`ere de voter rassemble de nombreux avantages qui justifient l'utilisation de ce syst\`eme de vote par rapport au vote majoritaire.



\subsection{Inconv\'enients}

Comme tous les syst\`emes de vote, il comporte \'egalement plusieurs inconv\'enients que nous ne jugeons pas probl\'ematique pour la primaire.

\end{document}


