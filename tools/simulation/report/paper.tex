\documentclass[conference]{IEEEtran}
\newcommand*{\rootPath}{./}

\usepackage{url}

% math and cs
% \usepackage[]{algorithm2e}
\usepackage[linesnumbered,lined,boxed,commentsnumbered]{algorithm2e}
\usepackage{amsmath}
\usepackage{amssymb}
\usepackage{mathrsfs}
\newtheorem{definition}{Definition}
\newtheorem{proposition}{Proposition}
\newenvironment{proof}[1][Proof]{\begin{trivlist}
  \item[\hskip \labelsep {\bfseries #1}]}{\end{trivlist}}
  % \newenvironment{definition}[1][Definition]{\begin{trivlist}
  % \item[\hskip \labelsep {\bfseries #1}]}{\end{trivlist}}
  % \newenvironment{example}[1][Example]{\begin{trivlist}
  % \item[\hskip \labelsep {\bfseries #1}]}{\end{trivlist}}
  % \newenvironment{remark}[1][Remark]{\begin{trivlist}
  % \item[\hskip \labelsep {\bfseries #1}]}{\end{trivlist}}

% style
\usepackage{booktabs}
\usepackage{multirow}
\usepackage{lipsum}
\usepackage{todonotes}
\usepackage{standalone}
\usepackage{import}
\usepackage{url}
%\Urlmuskip=0mu plus 1mu %bug url

% graph
\usepackage{graphicx}
\usepackage[outdir=./]{epstopdf}
\usepackage[labelformat=simple]{subcaption}
\usepackage{array}
%\usepackage[colorinlistoftodos]{todonotes}
\newcommand{\HRule}{\rule{\linewidth}{0.5mm}}



\DeclareCaptionLabelSeparator{periodspace}{.\quad}
\captionsetup{font=footnotesize,labelsep=periodspace,singlelinecheck=false}
\captionsetup[sub]{font=footnotesize,singlelinecheck=true}


\usepackage[english,american]{babel}



\usepackage[capitalise,nameinlink]{cleveref}
%Nice formats for \cref
\crefname{section}{Sect.}{Sect.}
\Crefname{section}{Section}{Sections}
\crefname{figure}{Fig.}{Fig.}
\Crefname{figure}{Figure}{Figures}

\usepackage{xspace}
%\newcommand{\eg}{e.\,g.\xspace}
%\newcommand{\ie}{i.\,e.\xspace}
\newcommand{\eg}{e.\,g.,\ }
\newcommand{\ie}{i.\,e.,\ }


\renewcommand\thesubfigure{(\alph{subfigure})}


%invert table
\usepackage{collcell}
\usepackage{datatool}
\usepackage{environ}

%\newcolumntype{C}[1]{>{\centering}m{#1}}
%\newcolumntype{C}[1]{>{\centering\raggedright\arraybackslash}m{#1}}
\newcolumntype{L}[1]{>{\raggedright\let\newline\\\arraybackslash\hspace{0pt}}m{#1}}
\newcolumntype{C}[1]{>{\centering\let\newline\\\arraybackslash\hspace{0pt}}m{#1}}
\newcolumntype{R}[1]{>{\raggedleft\let\newline\\\arraybackslash\hspace{0pt}}m{#1}}
 
\standalonetrue

\title{Syst\`eme de vote sur la Primaire.org}
 
\author{
    \IEEEauthorblockN{Pierre-Louis Guhur, Thibauld Favre}
}


\begin{document}
  
  
\maketitle


\begin{abstract} %% max 150 words
LaPrimaire.org organise une primaire pr\'esidentielle en France pour les \'elections de 2017. Elle rel\`eve le d\'efi de faire un vote d\'emocratique avec plus d'une centaine de candidats ! Pour des questions pratiques, chaque \'electeur ne peut pas \'etudier attentivement chacun des candidats. De plus, le syst\'eme de vote utilis\'e pour les pr\'esidentielles comporte plusieurs inconv\'enients qui faussent la repr\'esentativit\'e du vote. LaPrimaire.org a donc developp\'e un nouveau syst\'eme de vote, bas\'e sur le Jugement Majoritaire. 
En se basant sur des simulations de vote, nous avons estim\'e le nombre minimum d'\'electeurs n\'ecessaires pour valider la repr\'esentativit\'e et l'\'equir\'epartie de  notre syst\'eme de vote. 
\end{abstract}



\documentclass[conference]{IEEEtran}
\newcommand*{\rootPath}{../}
\usepackage{url}

% math and cs
% \usepackage[]{algorithm2e}
\usepackage[linesnumbered,lined,boxed,commentsnumbered]{algorithm2e}
\usepackage{amsmath}
\usepackage{amssymb}
\usepackage{mathrsfs}
\newtheorem{definition}{Definition}
\newtheorem{proposition}{Proposition}
\newenvironment{proof}[1][Proof]{\begin{trivlist}
  \item[\hskip \labelsep {\bfseries #1}]}{\end{trivlist}}
  % \newenvironment{definition}[1][Definition]{\begin{trivlist}
  % \item[\hskip \labelsep {\bfseries #1}]}{\end{trivlist}}
  % \newenvironment{example}[1][Example]{\begin{trivlist}
  % \item[\hskip \labelsep {\bfseries #1}]}{\end{trivlist}}
  % \newenvironment{remark}[1][Remark]{\begin{trivlist}
  % \item[\hskip \labelsep {\bfseries #1}]}{\end{trivlist}}

% style
\usepackage{booktabs}
\usepackage{multirow}
\usepackage{lipsum}
\usepackage{todonotes}
\usepackage{standalone}
\usepackage{import}
\usepackage{url}
%\Urlmuskip=0mu plus 1mu %bug url

% graph
\usepackage{graphicx}
\usepackage[outdir=./]{epstopdf}
\usepackage[labelformat=simple]{subcaption}
\usepackage{array}
%\usepackage[colorinlistoftodos]{todonotes}
\newcommand{\HRule}{\rule{\linewidth}{0.5mm}}



\DeclareCaptionLabelSeparator{periodspace}{.\quad}
\captionsetup{font=footnotesize,labelsep=periodspace,singlelinecheck=false}
\captionsetup[sub]{font=footnotesize,singlelinecheck=true}


\usepackage[english,american]{babel}



\usepackage[capitalise,nameinlink]{cleveref}
%Nice formats for \cref
\crefname{section}{Sect.}{Sect.}
\Crefname{section}{Section}{Sections}
\crefname{figure}{Fig.}{Fig.}
\Crefname{figure}{Figure}{Figures}

\usepackage{xspace}
%\newcommand{\eg}{e.\,g.\xspace}
%\newcommand{\ie}{i.\,e.\xspace}
\newcommand{\eg}{e.\,g.,\ }
\newcommand{\ie}{i.\,e.,\ }


\renewcommand\thesubfigure{(\alph{subfigure})}


%invert table
\usepackage{collcell}
\usepackage{datatool}
\usepackage{environ}


\standalonetrue

\begin{document}

%%=============================================================================



\section{Introduction}
\label{sec:intro}

Les fran\c{c}ais sont majoritairement d\'efiants envers leurs politiques. 
Une \'etude men\'e par CEVIPOF \cite{cevipof2016} montre qu'en janvier 2016, $67\%$ des fran\c{c}ais pensent que la d\'emocratie ne fonctionne pas tr\`es bien voire pire. $76\%$ des fran\c{c}ais disent que les \'elu(e)s et les dirigeant(e)s politiques fran\c{c}ais sont plut\^ot corrompu(e)s. 

LaPrimaire.org est une solution citoyenne pour r\'epondre \`a cette d\'efiance. 
Notre m\'ethode est d'am\'eliorer la repr\'esentativit\'e de la soci\'et\'e fran\c{c}aise. Pour cela, nous organisons une primaire pr\'esidentielle pour les \'elections de 2017 en France dans laquelle chaque citoyen fran\c{c}ais peut candidater. L'\'elu(e) recevra un soutien financier, l\'egal et les 500 signatures d'\'elus provenant de 30 d\'epartements diff\'erents avec 50 \'elus maximum par d\'epartement.

Pour faciliter l'apparition de nouveaux repr\'esentants, la Primaire repose sur plusieurs principes, comme indiqu\'es dans son manifeste~\cite{manifeste}: elle n'est pas un parti politique ; elle n'a pas de programme politique ; un candidat n'est pas jug\'e sur sa notori\'et\'e mais sur ses id\'ees.

Le syst\`eme de vote actuellement utilis\'e pour les \'elections pr\'esidentielles, appel\'e \emph{vote majoritaire}, va au contraire du principe de notori\'et\'e. Par exemple, il serait facile pour un chanteur de demander \'a ses fans de s'inscrire sur laPrimaire.org. Il disposerait alors d'un plus grand nombre de votes qu'un citoyen ordinaire. Le vote majoritaire souffre d'autres inconv\'enients qui d\'efavorisent la repr\'esentativit\'e, comme present\'e dans la \cref{sec:mv}. 

C'est pour cela que nous nous sommes tourn\'es vers un nouveau syst\`eme de vote, appel\'e jugement majoritaire (JM). Le principe de ce syst\`eme est de juger chaque candidat selon un bar\^eme de mentions: tr\`es bien, bien, assez bien, correct et d\'evaforable. Le gagnant d'une \'election est alors celui qui rassemble les meilleurs mentions au sens de la m\'edianne.  Le JM a \'et\'e propos\'e pour la premi\'ere fois par deux chercheurs fran\c{c}ais, Michel Balinski et Rida Laraki \cite{balinski2010majority}.  Il a \'et\'e test\'e lors des \'elections pr\'esidentielles de 2012 par un sondage command\'e par le \emph{think tank} Terra Nova  et r\'ealis\'e par l'institut OpinionWay aupr\`es d'un \'echantillon repr\'esentatif de 1034 personnes \cite{balinski2012rendre}. Dans la \cref{sec:mj}, nous r\'esumons ses avantages et inconv\'enients par rapport \`a d'autres syst\`emes de vote. 

Si ce syst\`eme permet d'am\'eliorer la repr\'esentativit\'e du vote, il n'est pas con\c{c}u pour fonctionner avec un grand nombre de candidats. Tous les \'electeurs ne peuvent pas \'etudier de mani\`ere approfondie plus de 100 candidats. Par soucis pratique, les candidats ayant de plus de notori\'et\'e m\'ediatique seraient encore avantag\'es. C'est pourquoi, nous avons deriv\'e le JM pour l'adapter \`a un grand nombre de candidats. Plus concr\`etement,
le vote se d\'eroule en deux phases. Au cours de la premi\`ere phase, chaque \'electeur s'exprime seulement sur un lot de candidats pour qualifier les meilleurs candidats. Lors de la seconde phase, les \'electeurs s'expriment sur ces candidats qualifi\'es.
Dans la \cref{sec:laprimaire}, nous montrons que comment notre proc\'ed\'e pour construire un lot permet d'assurer l'\'equir\'epartie entre les candidats. En se basant sur les r\'esultats du sondage OpinionWay/Terra Nova, nous avons extrapol\'e des r\'esultats de vote sur un nombre variable de candidats, puis simul\'e des \'elections. Cela a permis d'\'etablir le nombre minimum d'\'electeurs n\'ecessaire \`a rendre le vote fiable selon le nombre de candidats.



\end{document}




\documentclass[conference]{IEEEtran}
\newcommand*{\rootPath}{../}
\usepackage{url}

% math and cs
% \usepackage[]{algorithm2e}
\usepackage[linesnumbered,lined,boxed,commentsnumbered]{algorithm2e}
\usepackage{amsmath}
\usepackage{amssymb}
\usepackage{mathrsfs}
\newtheorem{definition}{Definition}
\newtheorem{proposition}{Proposition}
\newenvironment{proof}[1][Proof]{\begin{trivlist}
  \item[\hskip \labelsep {\bfseries #1}]}{\end{trivlist}}
  % \newenvironment{definition}[1][Definition]{\begin{trivlist}
  % \item[\hskip \labelsep {\bfseries #1}]}{\end{trivlist}}
  % \newenvironment{example}[1][Example]{\begin{trivlist}
  % \item[\hskip \labelsep {\bfseries #1}]}{\end{trivlist}}
  % \newenvironment{remark}[1][Remark]{\begin{trivlist}
  % \item[\hskip \labelsep {\bfseries #1}]}{\end{trivlist}}

% style
\usepackage{booktabs}
\usepackage{multirow}
\usepackage{lipsum}
\usepackage{todonotes}
\usepackage{standalone}
\usepackage{import}
\usepackage{url}
%\Urlmuskip=0mu plus 1mu %bug url

% graph
\usepackage{graphicx}
\usepackage[outdir=./]{epstopdf}
\usepackage[labelformat=simple]{subcaption}
\usepackage{array}
%\usepackage[colorinlistoftodos]{todonotes}
\newcommand{\HRule}{\rule{\linewidth}{0.5mm}}



\DeclareCaptionLabelSeparator{periodspace}{.\quad}
\captionsetup{font=footnotesize,labelsep=periodspace,singlelinecheck=false}
\captionsetup[sub]{font=footnotesize,singlelinecheck=true}


\usepackage[english,american]{babel}



\usepackage[capitalise,nameinlink]{cleveref}
%Nice formats for \cref
\crefname{section}{Sect.}{Sect.}
\Crefname{section}{Section}{Sections}
\crefname{figure}{Fig.}{Fig.}
\Crefname{figure}{Figure}{Figures}

\usepackage{xspace}
%\newcommand{\eg}{e.\,g.\xspace}
%\newcommand{\ie}{i.\,e.\xspace}
\newcommand{\eg}{e.\,g.,\ }
\newcommand{\ie}{i.\,e.,\ }


\renewcommand\thesubfigure{(\alph{subfigure})}


%invert table
\usepackage{collcell}
\usepackage{datatool}
\usepackage{environ}


\standalonetrue

\begin{document}

%%=============================================================================



\section{Le vote majoritaire}
\label{sec:mv}

\subsection{Le paradoxe de Condorcet}

\subsection{Les strat\'egies partisannes}

\end{document}



\documentclass[conference]{IEEEtran}
\newcommand*{\rootPath}{../}
\usepackage{url}

% math and cs
% \usepackage[]{algorithm2e}
\usepackage[linesnumbered,lined,boxed,commentsnumbered]{algorithm2e}
\usepackage{amsmath}
\usepackage{amssymb}
\usepackage{mathrsfs}
\newtheorem{definition}{Definition}
\newtheorem{proposition}{Proposition}
\newenvironment{proof}[1][Proof]{\begin{trivlist}
  \item[\hskip \labelsep {\bfseries #1}]}{\end{trivlist}}
  % \newenvironment{definition}[1][Definition]{\begin{trivlist}
  % \item[\hskip \labelsep {\bfseries #1}]}{\end{trivlist}}
  % \newenvironment{example}[1][Example]{\begin{trivlist}
  % \item[\hskip \labelsep {\bfseries #1}]}{\end{trivlist}}
  % \newenvironment{remark}[1][Remark]{\begin{trivlist}
  % \item[\hskip \labelsep {\bfseries #1}]}{\end{trivlist}}

% style
\usepackage{booktabs}
\usepackage{multirow}
\usepackage{lipsum}
\usepackage{todonotes}
\usepackage{standalone}
\usepackage{import}
\usepackage{url}
%\Urlmuskip=0mu plus 1mu %bug url

% graph
\usepackage{graphicx}
\usepackage[outdir=./]{epstopdf}
\usepackage[labelformat=simple]{subcaption}
\usepackage{array}
%\usepackage[colorinlistoftodos]{todonotes}
\newcommand{\HRule}{\rule{\linewidth}{0.5mm}}



\DeclareCaptionLabelSeparator{periodspace}{.\quad}
\captionsetup{font=footnotesize,labelsep=periodspace,singlelinecheck=false}
\captionsetup[sub]{font=footnotesize,singlelinecheck=true}


\usepackage[english,american]{babel}



\usepackage[capitalise,nameinlink]{cleveref}
%Nice formats for \cref
\crefname{section}{Sect.}{Sect.}
\Crefname{section}{Section}{Sections}
\crefname{figure}{Fig.}{Fig.}
\Crefname{figure}{Figure}{Figures}

\usepackage{xspace}
%\newcommand{\eg}{e.\,g.\xspace}
%\newcommand{\ie}{i.\,e.\xspace}
\newcommand{\eg}{e.\,g.,\ }
\newcommand{\ie}{i.\,e.,\ }


\renewcommand\thesubfigure{(\alph{subfigure})}


%invert table
\usepackage{collcell}
\usepackage{datatool}
\usepackage{environ}


\standalonetrue

\begin{document}

%%=============================================================================



\section{Le jugement majoritaire}
\label{sec:mj}


\end{document}




\documentclass[conference]{IEEEtran}
\newcommand*{\rootPath}{../}
\usepackage{url}

% math and cs
% \usepackage[]{algorithm2e}
\usepackage[linesnumbered,lined,boxed,commentsnumbered]{algorithm2e}
\usepackage{amsmath}
\usepackage{amssymb}
\usepackage{mathrsfs}
\newtheorem{definition}{Definition}
\newtheorem{proposition}{Proposition}
\newenvironment{proof}[1][Proof]{\begin{trivlist}
  \item[\hskip \labelsep {\bfseries #1}]}{\end{trivlist}}
  % \newenvironment{definition}[1][Definition]{\begin{trivlist}
  % \item[\hskip \labelsep {\bfseries #1}]}{\end{trivlist}}
  % \newenvironment{example}[1][Example]{\begin{trivlist}
  % \item[\hskip \labelsep {\bfseries #1}]}{\end{trivlist}}
  % \newenvironment{remark}[1][Remark]{\begin{trivlist}
  % \item[\hskip \labelsep {\bfseries #1}]}{\end{trivlist}}

% style
\usepackage{booktabs}
\usepackage{multirow}
\usepackage{lipsum}
\usepackage{todonotes}
\usepackage{standalone}
\usepackage{import}
\usepackage{url}
%\Urlmuskip=0mu plus 1mu %bug url

% graph
\usepackage{graphicx}
\usepackage[outdir=./]{epstopdf}
\usepackage[labelformat=simple]{subcaption}
\usepackage{array}
%\usepackage[colorinlistoftodos]{todonotes}
\newcommand{\HRule}{\rule{\linewidth}{0.5mm}}



\DeclareCaptionLabelSeparator{periodspace}{.\quad}
\captionsetup{font=footnotesize,labelsep=periodspace,singlelinecheck=false}
\captionsetup[sub]{font=footnotesize,singlelinecheck=true}


\usepackage[english,american]{babel}



\usepackage[capitalise,nameinlink]{cleveref}
%Nice formats for \cref
\crefname{section}{Sect.}{Sect.}
\Crefname{section}{Section}{Sections}
\crefname{figure}{Fig.}{Fig.}
\Crefname{figure}{Figure}{Figures}

\usepackage{xspace}
%\newcommand{\eg}{e.\,g.\xspace}
%\newcommand{\ie}{i.\,e.\xspace}
\newcommand{\eg}{e.\,g.,\ }
\newcommand{\ie}{i.\,e.,\ }


\renewcommand\thesubfigure{(\alph{subfigure})}


%invert table
\usepackage{collcell}
\usepackage{datatool}
\usepackage{environ}


\standalonetrue

\begin{document}

%%=============================================================================




\section{Notre syst\`eme de vote}
\label{sec:laprimaire}

\end{document}





\section{Conclusion}





%%=============================================================================
%%=============================================================================
\ifstandalone
	\bibliographystyle{IEEEtran}
	\bibliography{\rootPath Annexes/biblio.bib}
\fi
%%=============================================================================
%%=============================================================================

\end{document}
